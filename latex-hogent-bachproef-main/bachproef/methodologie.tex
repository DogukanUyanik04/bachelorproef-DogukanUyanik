%%=============================================================================
%% Methodologie
%%=============================================================================

\chapter{\IfLanguageName{dutch}{Methodologie}{Methodology}}%
\label{ch:methodologie}

%% TODO: In dit hoofstuk geef je een korte toelichting over hoe je te werk bent
%% gegaan. Verdeel je onderzoek in grote fasen, en licht in elke fase toe wat
%% de doelstelling was, welke deliverables daar uit gekomen zijn, en welke
%% onderzoeksmethoden je daarbij toegepast hebt. Verantwoord waarom je
%% op deze manier te werk gegaan bent.
%% 
%% Voorbeelden van zulke fasen zijn: literatuurstudie, opstellen van een
%% requirements-analyse, opstellen long-list (bij vergelijkende studie),
%% selectie van geschikte tools (bij vergelijkende studie, "short-list"),
%% opzetten testopstelling/PoC, uitvoeren testen en verzamelen
%% van resultaten, analyse van resultaten, ...
%%
%% !!!!! LET OP !!!!!
%%
%% Het is uitdrukkelijk NIET de bedoeling dat je het grootste deel van de corpus
%% van je bachelorproef in dit hoofstuk verwerkt! Dit hoofdstuk is eerder een
%% kort overzicht van je plan van aanpak.
%%
%% Maak voor elke fase (behalve het literatuuronderzoek) een NIEUW HOOFDSTUK aan
%% en geef het een gepaste titel.


Om de onderzoeksvraag te beantwoorden, hanteert dit onderzoek een methodiek gericht op de ontwikkeling van een Proof of Concept (PoC), gevolgd door een experimentele validatiefase.
Vanwege de complexiteit van de hardware-integratie wordt een iteratieve, agile aanpak gehanteerd waarin risicomanagement centraal staat. 

Het onderzoek is onderverdeeld in de volgende vier-fasen:

\subsection{Fase 1: Requirements Analyse en Stakeholder-validatie}
\label{subsec:fase1-requirements}

In de eerste fase van het onderzoek worden de functionele en niet-functionele vereisten van de Mixed Reality-applicatie gedefinieerd. Het doel is om een objectief kader te scheppen waaraan de \textit{Proof of Concept} (PoC) moet voldoen om de geidentificeerde problemen, zoals het \textit{split-attention effect}, effectief aan te pakken.

\begin{itemize}
    \item \textbf{Doelstelling:} Het vastleggen van de technische en medische randvoorwaarden voor een effectieve reanimatietraining in MR.
    \item \textbf{Onderzoeksmethoden:}
    \begin{itemize}
        \item \textit{Literatuurstudie:} Analyse van de ERC 2025-richtlijnen voor kwalitatieve borstcompressies \autocite{Smyth2025, Olasveengen2021}.
        \item \textit{Stakeholder-consultatie:} Een verkennend gesprek en formele validatie van het onderzoeksvoorstel door de opdrachtgever (Zorglab), waarbij specifieke aandachtspunten zoals handpositie en beademingstechnieken zijn geïdentificeerd.
        \item \textit{MoSCoW-prioritering:} Het rangschikken van de vereisten (\textit{Must have, Should have, Could have, Won't have}) om de scope van de PoC af te bakenen.
    \end{itemize}
    \item \textbf{Deliverables:} Een geprioriteerde lijst van functionele vereisten (o.a. real-time diepte- en frequentiemeting) en niet-functionele vereisten (o.a. latentie onder de 100 ms).
    \item \textbf{Verantwoording:} Deze fase is essentieel om te garanderen dat de ontwikkelde oplossing aansluit bij de behoeften van de medische sector en dat de technische realisatie gericht is op de meest kritieke leerparameters \autocite{Sigrist2013}.
\end{itemize}


\subsection{}
\subsection{}
\subsection{}



