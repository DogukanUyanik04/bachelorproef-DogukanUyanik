%%=============================================================================
%% Inleiding
%%=============================================================================

\chapter{\IfLanguageName{dutch}{Inleiding}{Introduction}}%
\label{ch:inleiding}

\section{\IfLanguageName{dutch}{Context en urgentie}{Context and Urgency}}%
\label{sec:context_en_urgentie}

Ondanks de enorme technologische vooruitgang gemaakt in de moderne geneeskunde de afgelopen decennia, blijven hart- en 
vaatziekten de absolute nummer één doodsoorzaak wereldwijd. Jaarlijks is deze groep aandoeningen verantwoordelijk voor ongeveer 
een derde van alle sterfgevallen, wat neerkomt op miljoenen levens die vroegtijdig eindigen \autocite{owid-causes-of-death}. 
Wanneer een hartstilstand buiten het ziekenhuis plaatsvindt, krimpt de marge tussen leven en dood tot enkele kritische minuten. 
In deze fractie van tijd is de overlevingskans van het slachtoffer bijna volledig afhankelijk van één variabele: de kwaliteit van de 
omstanders of zorgverleners toegepaste reanimatie (CPR).

Hoewel het belang van kwaliteitsvolle reanimatie breed erkend wordt, blijft het effectief aanleren en onderhouden van deze motorische 
vaardigheid een grote uitdaging. De effectiviteit van een hulpverlener hangt af van uiterst precieze parameters, zoals een specifieke 
compressiediepte en een constant ritme. Het is net in deze educatieve fase, waar studenten de vertaalslag maken van theorie naar fysieke 
handeling, dat technologie een cruciale rol speelt.

\section{\IfLanguageName{dutch}{Probleemstelling}{Problem Statement}}%
\label{sec:probleemstelling}

De doelgroep van dit onderzoek is specifiek afgebakend tot bachelorstudenten in de opleiding Verpleegkunde en Geneeskunde aan 
HOGENT en partners zoals het UZ Gent en Zorglab. Voor deze toekomstige zorgverleners is het behalen van een hoge graad van bekwaamheid 
in CPR-technieken een essentieel onderdeel van het curriculum. In huidige onderwijsmethodiek wordt voor deze training vaak gebruikgemaakt
 van digitale oefenpoppen, zoals de Laerdal QCPR-serie. Deze poppen registreren via interne sensoren de kwaliteit van de handelingen, 
 maar de wijze waarop deze informatie aan de student wordt teruggekoppeld vertoont enkele tekortkomingen.

Ten eerste is er de barrière van abstracte feedback. Huidige systemen vertalen fysieke compressies naar numerieke scores, 
percentages of 2D-grafieken op een extern scherm, zoals een iPad of tablet. Voor een student die een motorische vaardigheid 
aanleert is deze vorm van feedback erg abstract en ingewikkeld te vertalen naar een concrete fysieke correctie \textcite{Sigrist2013}. 
De student ziet op het scherm wellicht een score van 60\%, maar krijgt geen directe, ruimtelijke instructie over hoe de handpositie of 
de drukhoek in de driedimensionale ruimte moet worden aangepast zodat die score verhoogd kan worden. 

Ten tweede veroorzaakt het gebruik van een extern scherm het zogenaamde split-attention effect \textcite{Sigrist2013}. Om de feedback te ontvangen wordt de student genoodzaakt de focus te verschuiven van de ‘patiënt’ (de oefenpop) naar de tablet. Dit verbreekt niet alleen de concentratie en de flow van de handeling, maar is ook niet representatief voor een realistische noodsituatie. In een werkelijke reanimatiesetting is een onafgebroken focus op het slachtoffer immer van levensbelang.

Deze problematiek staat in schril contrast met de bredere technologische evolutie binnen de gezondheidszorg. Terwijl Mixed Reality (MR) en Augmented Reality (AR) reeds succesvol worden ingezet voor chirurgische visualisaties an anatomische lessen \autocite{DAngelo2021}, blijft reanimatietraining vooralsnog achter. Het verrijken van de werkelijkheid door feedback direct over de oefenpop te projecteren, biedt een logische oplossing om de hierboven beschreven cognitieve beperkingen weg te nemen.

\section{\IfLanguageName{dutch}{Onderzoeksvraag}{Research question}}%
\label{sec:onderzoeksvraag}

Op basis van bovenstaande probleemstelling wordt de volgende centrale onderzoeksvraag geformuleerd: “In welke mate kan een Mixed Reality-applicatie de kwaliteit van feedback tijdens reanimatietraining verbeteren in vergelijking met de huidige tablet-gebaseerde methoden?”

Om een onderbouwd antwoord te geven op deze hoofdvraag, wordt het onderzoek gesplitst in meerdere deelaspecten: 
\begin{itemize}
    \item \textbf{Probleemdomein:} Welke veelvoorkomende reanimatiefouten worden door de huidige hardware onvoldoende gevisualiseerd? En hoe ervaren studenten de vertaalslag van abstracte data naar fysieke handeling?
    \item \textbf{Oplossingsdomein:} Welke visualisatietechnieken binnen Mixed Reality zijn het meest effectief voor het aanleren van motorische vaardigheden? Hoe kan sensordata van de oefenpop real-time verwerkt worden binnen een MR-omgeving?
\end{itemize}

\section{\IfLanguageName{dutch}{Onderzoeksdoelstelling}{Research objective}}%
\label{sec:onderzoeksdoelstelling}

Het doel van dit onderzoek is de ontwikkeling en validatie van een Proof of Concept (PoC) om de potentie van MR binnen reanimatietraining te staven. Het concrete resultaat bestaat uit twee delen:
\begin{itemize}
    \item Een werkend prototype ontwikkeld in Unity voor de Meta Quest 3, dat via Bluetooth communiceert met de oefenpop en visuele feedback (zoals de Ghost Avatar) projecteert.
    \item Een validatierapport waarin de effectiviteit van deze oplossing wordt vergeleken met traditionele methoden, gebruikmakend van objectieve prestatiescores en de NASA-TLX schaal voor cognitieve belasting.
\end{itemize}

\section{\IfLanguageName{dutch}{Opzet van deze bachelorproef}{Structure of this bachelor thesis}}%
\label{sec:opzet-bachelorproef}

De rest van deze bachelorproef is als volgt opgebouwd:

In Hoofdstuk~\ref{ch:stand-van-zaken} wordt een overzicht gegeven van de stand van zaken binnen het onderzoeksdomein op basis van een literatuurstudie.

In Hoofdstuk~\ref{ch:methodologie} wordt de methodologie toegelicht en worden de technische stappen voor de ontwikkeling van de PoC besproken om een antwoord te kunnen formuleren op de onderzoeksvragen.

In Hoofdstuk 4 (Resultaten) behandelt de data-analyse van de experimenten, waarna in Hoofdstuk~\ref{ch:conclusie}, tenslotte, de conclusie wordt gegeven en een antwoord geformuleerd op de onderzoeksvragen. Daarbij wordt ook een aanzet gegeven voor toekomstig onderzoek binnen dit domein.