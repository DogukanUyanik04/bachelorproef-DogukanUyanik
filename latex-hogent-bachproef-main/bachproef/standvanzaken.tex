\chapter{\IfLanguageName{dutch}{Stand van zaken}{State of the art}}%
\label{ch:stand-van-zaken}

Dit hoofdstuk biedt een diepgaande analyse van de theoretische en technische fundamenten die noodzakelijk zijn voor de ontwikkeling van dit onderzoek. 
Waar de inleiding de maatschappelijke relevantie van reanimatietraining schetste, spitst dit hoofdstuk zich toe op de medische standaarden, 
de psychologische aspecten van motorisch leren en de technologische mogelijkheden van Mixed Reality. Deze literatuurstudie vormt de wetenschappelijke 
basis voor de methodologie en de technische keuzes die in \autoref{ch:methodologie} worden toegelicht.

\section{Reanimatie: Richtlijnen en Kwaliteitscriteria}
\label{sec:reanimatie-richtlijnen}

Om een effectief feedback-systeem te ontwerpen, moet eerst worden vastgesteld aan welke medische standaarden een kwalitatieve reanimatie moet voldoen. 
De leidraad hiervoor wordt gevormd door de internationale protocollen van de European Resuscitation Council (ERC).

\subsection{De ERC Guidelines 2025}
\label{subsec:erc-guidelines-2025}

De meest recente standaarden voor reanimatie zijn vastgelegd in de \textit{ERC Guidelines 2025 for Adult Basic Life Support}. 
Deze richtlijnen zijn gebaseerd op de wetenschappelijke consensus van de International Liaison Committee on Resuscitation (ILCOR) en 
introduceren enkele cruciale verschuivingen ten opzichte van de voorgaande 2021-editie \autocite{Smyth2025}.

De 2025-richtlijnen leggen een verhoogde nadruk op de snelheid van handelen en de ondersteunende rol van technologie in de overlevingsketen.
 De belangrijkste wijzigingen en aandachtspunten zijn:

\begin{itemize}
    \item \textbf{Onmiddellijk alarmeren (Call First):} In tegenstelling tot eerdere protocollen wordt nu geadviseerd om de noodcentrale direct te bellen bij een niet-reageerbaar slachtoffer, nog vóór de volledige beoordeling van de ademhaling. De ademhalingscontrole vindt plaats terwijl de hulpverlener de dispatcher aan de lijn heeft \autocite{Smyth2025}.
    \item \textbf{Herkenning van agonale ademhaling:} Er is een grotere focus op het herkennen van abnormale ademhalingspatronen zoals "gasping" (happen naar lucht) en "panting" (bij sporters), die vaak foutief als een teken van leven worden geïnterpreteerd \autocite{Smyth2025}.
    \item \textbf{Integratie van technologie:} Voor het eerst worden digitale hulpmiddelen, variërend van video-instructies door dispatchers tot het gebruik van Virtual Reality (VR) en Augmented Reality (AR) in trainingen, expliciet ingebed in de relevante secties van de richtlijnen in plaats van als aparte categorie \autocite{Smyth2025}.
\end{itemize}

Deze verschuiving naar technologische integratie legitimeert de inzet van Mixed Reality als trainingsinstrument. 
Het systeem moet de gebruiker niet alleen instrueren over de techniek, maar ook ondersteunen in de kritieke besluitvorming die door de ERC 2025 wordt benadrukt.

\subsection{Kwaliteitscriteria voor High-Quality CPR (HQ-CPR)}
\label{subsec:kwaliteitscriteria-cpr}

De effectiviteit van reanimatie wordt direct bepaald door de fysiologische impact van de handelingen op het lichaam van het slachtoffer. Om de overlevingskans te maximaliseren, 
hanteert de \textcite{Smyth2025} strikte kwantitatieve criteria voor wat wordt gedefinieerd als 'High-Quality CPR'. 
Voor de ontwikkeling van een Mixed Reality feedback-systeem zijn deze parameters de kernwaarden waarop de algoritmen en visuele cues worden gekalibreerd.

De belangrijkste technische parameters zijn samengevat in \autoref{tab:cpr-metrics}:

\begin{table}[h]
    \centering
    \begin{tabular}{lll}
        \toprule
        \textbf{Parameter} & \textbf{Doelwaarde} & \textbf{Fysiologisch belang} \\
        \midrule
        Compressiediepte & 5 -- 6 cm & Genereren van voldoende bloedstroom \\
        Compressiefrequentie & 100 -- 120 bpm & Optimaliseren van de coronaire perfusiedruk \\
        Borstkas-recoil & Volledige ontlasting & Toelaten van ventriculaire vulling \\
        Handpositie & Onderste helft sternum & Maximale compressie van de ventrikels \\
        \bottomrule
    \end{tabular}
    \caption{Technische parameters voor kwalitatieve reanimatie conform ERC 2025 \autocite{Smyth2025}.}
    \label{tab:cpr-metrics}
\end{table}

\paragraph{Compressiediepte}
De richtlijnen schrijven een diepte voor van minimaal 5 cm, maar niet meer dan 6 cm. Compressies die ondieper zijn dan 5 cm resulteren in een 
inadequate bloedstroom naar de hersenen en het hart. Omgekeerd vergroten compressies dieper dan 6 cm het risico op iatrogeen letsel, zoals ribfracturen of 
interne bloedingen \autocite{Smyth2025}. In een MR-omgeving is dit de meest uitdagende parameter om te visualiseren door het gebrek aan diepteperceptie van bovenaf.

\paragraph{Compressiefrequentie}
Het ideale ritme ligt tussen de 100 en 120 compressies per minuut. Een te lage frequentie leidt tot een te lage gemiddelde perfusiedruk, terwijl een te hoge 
frequentie (boven de 120 bpm) de borstkas onvoldoende tijd geeft voor recoil, wat de efficiëntie per compressie verlaagt.

\paragraph{Recoil en Leunen (Leaning)}
Een veelvoorkomende fout bij hulpverleners is het blijven 'leunen' op de borstkas na een compressie. De ERC benadrukt dat 
volledige \textit{recoil} (terugslag) essentieel is zodat het intrathoracale volume herstelt en het hart zich opnieuw met 
bloed kan vullen voor de volgende cyclus. Feedback-systemen moeten hier specifiek op controleren, aangezien dit een fout is 
die door de hulpverlener vaak niet wordt gevoeld.

\subsection{Manikin-gebaseerde Feedback: Laerdal QCPR}
\label{subsec:laerdal-qcpr}

De huidige standaard voor objectieve feedback in reanimatietraining wordt gevormd door intelligente oefenpoppen, waarvan de Laerdal 
QCPR-technologie de meest wijdverspreide is. Dit systeem maakt gebruik van een draadloze verbinding (vaak Bluetooth) tussen sensoren in 
de pop en externe software op een monitor of tablet \autocite{Cortegiani2017}.

\paragraph{Werking van de sensoren}
De sensoren in de pop meten continu de mechanische belasting op de borstkas. De belangrijkste gemeten parameters zijn de compressiediepte, 
de frequentie, de handpositie en de mate van volledige ontlasting (\textit{recoil}). Volgens \textcite{Cortegiani2017} berekent de software op 
basis van deze data een zogenaamde 'compression score': een gewogen gemiddelde (0--100\%) dat de algehele kwaliteit van de reanimatie uitdrukt. 
In hun onderzoek bij middelbare scholieren bleek dat studenten die trainden met deze real-time feedback een significant hogere score behaalden (mediaan 90\%) 
dan studenten die enkel vertrouwden op de instructies van een menselijke expert (mediaan 67\%) \autocite{Cortegiani2017}.

\begin{figure}[H] 
    \centering
    \includegraphics[width=0.8\textwidth]{../graphics/littleAnnCPR} 
    \caption{De Laerdal Little Anne QCPR pop in combinatie met de feedback-app op een tablet \autocite{Cortegiani2017}.}
    \label{fig:littleann-cpr}
\end{figure}

\paragraph{Superieure herkenning van technische fouten}
Een inzicht uit de studie van \textcite{Cortegiani2017} is dat softwarematige feedback vooral effectief is 
voor parameters die voor het menselijk oog moeilijk waarneembaar zijn. Zo presteerde de QCPR-groep significant beter op het 
gebied van \textit{chest recoil} (71\% correcte uitvoering versus 24\% in de controlegroep). Dit suggereert dat menselijke instructeurs 
moeite hebben om minieme fouten in de ontlasting van de borstkas te zien, terwijl de sensoren in de pop dit feilloos registreren. 

Hoewel dit systeem de technische vaardigheidsverwerving versnelt, blijft de interface beperkt tot een 2D-weergave op een extern scherm. 
Zoals in de volgende sectie wordt toegelicht, creëert dit een nieuwe uitdaging op het gebied van aandacht en cognitieve belasting.

\subsection{De beperkingen van 2D-displays en het Split-Attention Effect}
\label{subsec:2d-beperkingen}

Hoewel systemen zoals Laerdal QCPR objectieve data leveren, dwingen ze de gebruiker tot een onnatuurlijke werkhouding waarbij de aandacht verdeeld moet worden tussen de fysieke handeling en een extern scherm. Recent onderzoek van \textcite{Hurt2024} werpt een nieuw licht op de cognitieve nadelen van deze methode.

\paragraph{Situational Awareness en Performance}
Een cruciale bevinding in het onderzoek naar motorische taken is de sterke positieve correlatie tussen 
\textit{situational awareness} (SA) en prestatie bij het gebruik van Mixed Reality. \textcite{Hurt2024} stelt dat het 
beperken van \textit{split-attention} essentieel is: wanneer informatie direct in het gezichtsveld wordt geprojecteerd, 
kunnen deelnemers hun focus volledig op de taak houden, wat leidt tot betere resultaten. Opvallend genoeg ontbrak deze correlatie 
bij het gebruik van tablets. Dit suggereert dat de tablet-modaliteit de gebruiker dwingt om de aandacht zo onregelmatig te verdelen dat 
de natuurlijke link tussen omgevingsbewustzijn en prestatie wordt verbroken \autocite{Hurt2024}.

\paragraph{De Familiarity Bias en het TAM2-model}
Een belangrijk punt voor dit onderzoek is waarom tablets momenteel de standaard zijn, ondanks de cognitieve nadelen. 
\textcite{Hurt2024} verklaart dit aan de hand van het \textit{Technology Acceptance Model 2} (TAM2). Deelnemers rapporteren vaak een voorkeur voor tablets, 
niet omdat ze efficiënter zijn, maar vanwege een \textit{familiarity bias}. Omdat studenten in hun dagelijks leven constant tablets gebruiken voor studie en 
ontspanning, ervaren ze een lagere drempel in gebruiksgemak, ongeacht de werkelijke taakbelasting \autocite{Hurt2024}. 
Bij complexe, taakgerichte handelingen zoals CPR kan deze gewoonte echter de technische tekortkomingen van de modaliteit maskeren.

\paragraph{Task Load en Visuele Continuïteit}
Aanvullend onderzoek toont aan dat de \textit{task load} — een maatstaf voor mentale, fysieke en temporele inspanning — significant lager is bij 
Augmented Reality-instructies dan bij instructies op een monitor of tablet \autocite[Tang et al. in]{Hurt2024}. 
Door informatie via een \textit{heads-up display} (HUD) aan te bieden, hoeft de gebruiker zijn ogen vrijwel nooit van de taak af te wenden. 
Dit vermindert niet alleen de oog- en hoofdbewegingen, maar zorgt er ook voor dat informatie "in context" blijft \autocite[Haines et al. in]{Hurt2024}. 
Voor reanimatietraining betekent dit dat de overgang van het observeren van de feedback naar het uitvoeren van de compressie naadloos verloopt, in tegenstelling 
tot de gefragmenteerde ervaring bij 2D-schermen.


\subsection{De Guidance Hypothesis en de Ontwikkeling van Interne Representaties}
\label{subsec:guidance-hypothesis}

Een fundamenteel risico bij het gebruik van real-time feedbacksystemen is de \textit{guidance hypothesis}. 
Volgens \textcite{Sigrist2013} kan permanente externe feedback tijdens het leerproces leiden tot een afhankelijkheid 
van deze informatiebron, wat de ontwikkeling van een autonoom spiergeheugen belemmert.

\paragraph{Het mechanisme van afhankelijkheid}
De kern van de \textit{guidance hypothesis} is dat de student de externe feedback (bijv. een balkje op een scherm) gebruikt als een kruk. 
In plaats van te leren vertrouwen op \textit{intrinsic feedback}, de interne zintuiglijke waarneming van de eigen lichaamshouding en krachtoefening, 
reageert de student puur op de \textit{augmented feedback} van het systeem \autocite{Sigrist2013}. Dit heeft tot gevolg dat de prestaties tijdens de training 
uitstekend zijn, maar dat de vaardigheid onmiddellijk degradeert zodra de feedback wordt weggenomen (\textit{retention tests}). 
De student heeft namelijk geen "interne bewegingsrepresentatie" opgebouwd die zonder hulp kan worden opgeroepen.



\paragraph{Simpele versus complexe motorische taken}
\textcite{Sigrist2013} maken een belangrijk onderscheid tussen simpele en complexe taken. Bij simpele taken is constante feedback vrijwel altijd schadelijk voor het 
leerproces op lange termijn. Echter, bij complexe taken zoals reanimatie, waarbij houding, diepte, tempo en recoil simultaan moeten worden beheerst, 
kan real-time feedback juist helpen om de cognitieve belasting te verlagen in de vroege leerfase. Het helpt de student om de "structuur" van de beweging te begrijpen 
zonder overweldigd te raken.

\paragraph{Strategieën voor effectief leren}
Om de nadelen van de \textit{guidance hypothesis} te beperken, stelt de literatuur verschillende strategieën voor die relevant zijn voor het ontwerp van een 
Mixed Reality PoC:
\begin{itemize}
    \item \textbf{Fading feedback:} De frequentie van de visuele cues moet afnemen naarmate het vaardigheidsniveau van de student toeneemt \autocite{Sigrist2013}.
    \item \textbf{No-feedback trials:} Het inbouwen van oefensessies zonder hologrammen is essentieel om te controleren of de student de correcte diepte en frequentie op gevoel kan reproduceren.
    \item \textbf{Bandwidth feedback:} Hierbij wordt alleen feedback gegeven wanneer de foutmarge een bepaalde drempel overschrijdt, wat de student dwingt om binnen de 
    "veilige zone" zelfstandig te corrigeren.
\end{itemize}

Door de \textit{ghost avatar} in Mixed Reality niet constant, maar adaptief aan te bieden, kan het systeem de student begeleiden in de vroege fase zonder dat er een 
permanente afhankelijkheid ontstaat die de inzetbaarheid in een echte noodsituatie ondermijnt.


\section{Cognitieve Belasting en Motorisch Leren}
\label{sec:cognitieve-belasting}

In deze sectie wordt de psychologische rechtvaardiging voor de overstap naar Mixed Reality toegelicht. De effectiviteit van een trainingstool wordt niet alleen bepaald door de technische accuraatheid van de sensoren, maar vooral door de manier waarop de menselijke hersenen informatie verwerken tijdens een fysiek veeleisende taak zoals reanimatie.

\subsection{Het Split-Attention Effect}
\label{subsec:split-attention}

Het \textit{split-attention effect} treedt op wanneer een leerling gedwongen wordt om de aandacht te verdelen tussen twee of meer informatiebronnen die essentieel zijn voor de taak, maar die fysiek van elkaar gescheiden zijn. Volgens \textcite{Hurt2024} leidt dit bij traditionele CPR-trainingen met externe schermen tot een "cognitieve boete". 

De student moet herhaaldelijk de blik afwenden van de borstkas van de pop (de plaats van actie) naar een tablet of monitor (de plaats van feedback). Volgens de \textit{Cognitive Load Theory} zorgt dit voor een overbelasting van het werkgeheugen: de hersenen moeten de visuele context van de pop telkens mentaal loslaten om de abstracte data op het scherm te verwerken en vervolgens weer terug te koppelen naar de fysieke handeling. Mixed Reality elimineert deze barrière door de feedback direct over de pop te projecteren, wat de visuele continuïteit herstelt en de mentale verwerkingscapaciteit volledig beschikbaar maakt voor de motorische uitvoering.

\subsection{Situational Awareness (SA)}
\label{subsec:situational-awareness}

\textit{Situational Awareness} (SA) is het vermogen om elementen in de omgeving waar te nemen, de betekenis ervan te begrijpen en te voorspellen hoe deze de taak in de nabije toekomst zullen beïnvloeden. In een reële reanimatiesituatie is SA van vitaal belang om adequaat te reageren op de AED-instructies, de toestand van het slachtoffer en de veiligheid van de omgeving.

Uit het onderzoek van \textcite{Hurt2024} blijkt dat er bij Mixed Reality een significante positieve correlatie bestaat tussen SA en de uiteindelijke prestatie. 
Bij tablet-gebaseerde modaliteiten ontbrak dit verband echter. Gebruikers van externe schermen vertonen vaak een vorm van "tunnelvisie" op de grafische feedback, 
waardoor zij het contact met de pop en de omgeving verliezen. Hurt wijst hierbij ook op de \textit{familiarity bias}: omdat studenten zeer gewend zijn aan tablets, 
onderschatten zij vaak de hoeveelheid aandacht die het schakelen tussen scherm en pop werkelijk opeist.

\subsection{Proprioceptie en Haptiek}
\label{subsec:proprioceptie-haptiek}

Een uniek aspect van reanimatie is de noodzaak van haptische feedback: de fysieke weerstand van de borstkas die de hulpverlener moet voelen om de juiste kracht 
te kalibreren. \textcite{Sigrist2013} maken hierbij onderscheid tussen twee vormen van perceptie:

\begin{itemize}
    \item \textbf{Haptiek:} De combinatie van tactiele perceptie (druk op de huid) en kinesthetische perceptie (het gevoel van de pose en beweging via receptoren 
    in spieren en pezen).
    \item \textbf{Proprioceptie:} Het interne vermogen van het lichaam om de eigen positie en beweging in de ruimte waar te nemen zonder visuele controle.
\end{itemize}

Motorisch leren is volgens \textcite{Sigrist2013} het meest effectief wanneer externe feedback de interne zintuiglijke waarneming "kalibreert". In een traditionele 
setting voelen de handen de weerstand, maar de ogen zien een abstracte 2D-meter. In een Mixed Reality-omgeving vallen de visuele referentie (de \textit{ghost avatar}) 
en de haptische ervaring (de fysieke weerstand van de pop) ruimtelijk samen. Dit zorgt voor een snellere neurologische koppeling: de hersenen leren de gevoelde 
spierspanning direct te associëren met de correcte diepte van 5--6 cm.



\section{Mixed Reality in de Gezondheidszorg}
\label{sec:mr-gezondheidszorg}

In deze sectie worden de technologische kaders van Mixed Reality (MR) geschetst. 
Er wordt dieper ingegaan op de hardwarematige verschillen tussen huidige systemen en hoe deze van invloed zijn op de nauwkeurigheid van medische simulaties.

\subsection{Definities: Het Reality-Virtuality Continuum}
\label{subsec:definities-mr}

Om de positie van Mixed Reality te begrijpen, wordt vaak verwezen naar het \textit{Reality-Virtuality Continuum} van Milgram. 
Volgens \textcite{Gerup2020} is Mixed Reality de overkoepelende term voor omgevingen waarin fysieke en digitale objecten in real-time naast elkaar bestaan en met elkaar interageren. 

\begin{itemize}
    \item \textbf{Augmented Reality (AR):} Voegt digitale elementen toe aan een verder ongewijzigde fysieke wereld, vaak via een smartphone of transparante bril \autocite{Carroll2020}.
    \item \textbf{Virtual Reality (VR):} Sluit de gebruiker volledig af van de fysieke wereld en vervangt deze door een volledig gesimuleerde omgeving \autocite{Raymer2023}.
    \item \textbf{Mixed Reality (MR):} De samensmelting waarbij virtuele objecten (zoals de \textit{ghost avatar}) niet alleen bovenop de werkelijkheid worden getoond, maar ook verankerd lijken aan fysieke objecten (zoals de reanimatiepop) \autocite{DAngelo2021}.
\end{itemize}



\subsection{Video Passthrough versus Optical See-Through}
\label{subsec:passthrough-vs-optical}

Bij het ontwerpen van een reanimatietrainer is de keuze van de visuele weergave cruciaal. Er bestaan twee dominante technologieën:

\begin{enumerate}
    \item \textbf{Optical See-Through (bijv. Microsoft HoloLens 2):} Maakt gebruik van transparante glazen waar hologrammen op geprojecteerd worden. Het voordeel is een natuurlijke weergave van de omgeving, maar het nadeel is een beperkt gezichtsveld (\textit{Field of View}) en transparante hologrammen die bij fel licht moeilijk zichtbaar zijn \autocite{Diller2024}.
    \item \textbf{Video Passthrough (bijv. Meta Quest 3):} De headset is volledig gesloten (zoals VR), maar gebruikt hoge-resolutie camera's om de buitenwereld in real-time te filmen en op de schermen te tonen. 
\end{enumerate}

Voor dit onderzoek is gekozen voor de Meta Quest 3 vanwege de superieure \textbf{diepteperceptie} en \textbf{occlusie}. 
Bij reanimatie bevinden de handen zich tussen de ogen van de hulpverlener en de pop. Video Passthrough staat toe dat virtuele feedback 
(zoals diepte-indicators) correct achter of rondom de fysieke handen getoond kan worden, wat bij optische systemen vaak problematisch is \autocite{Glover2019}.



\subsection{Spatial Computing en Verankering (Anchoring)}
\label{subsec:spatial-anchoring}

Een essentieel aspect van de Proof of Concept is dat de \textit{ghost avatar} exact gepositioneerd blijft op de fysieke reanimatiepop, 
ongeacht de bewegingen van de gebruiker. Dit proces wordt gefaciliteerd door \textit{Spatial Computing}.

Volgens \textcite{Glover2019} maakt de headset gebruik van \textit{Simultaneous Localization and Mapping} (SLAM) om de kamer in kaart te brengen.
Door middel van \textbf{Spatial Anchors} kan de Unity-applicatie een digitaal coördinatensysteem koppelen aan een specifiek punt in de fysieke ruimte. 
Voor de reanimatiepop betekent dit dat de software de pop herkent als een vast ankerpunt, waardoor de feedback-overlay stabiel blijft en de student de 
haptische weerstand van de pop exact op de locatie van het hologram ervaart.





\section{Visualisatietechnieken voor Motorische Feedback}
\label{sec:visualisatietechnieken}

De effectiviteit van een Mixed Reality-reanimatietrainer hangt grotendeels af van de manier waarop de verzamelde sensordata 
wordt vertaald naar begrijpelijke visuele instructies. 
In deze sectie worden de verschillende technieken voor visuele ondersteuning besproken, gebaseerd op de classificatie van \textcite{Diller2024}.

\subsection{Positionele Feedback: De Ghost Avatar}
\label{subsec:positionele-feedback}

Positionele feedback heeft als doel de gebruiker te informeren over de gewenste lichaamshouding of de positie van de ledematen. 
De meest gebruikte techniek hiervoor is de \textit{Transparent Target Avatar}, ook wel bekend als de 'Ghost Avatar'.

Volgens de uitgebreide survey van \textcite{Diller2024} biedt een transparante avatar het grote voordeel dat het dient als een direct ruimtelijk spiegelbeeld. 
In de context van reanimatie kan een Ghost Avatar de ideale handpositie en de volledige compressiecyclus (van de startpositie tot de maximale diepte van 5--6 cm) 
visualiseren. Doordat het hologram direct over de eigen handen en de pop wordt geprojecteerd, hoeft de student geen mentale rotatie of 
vertaalslag te maken vanaf een extern 2D-scherm. Dit minimaliseert de cognitieve belasting en maximaliseert de nauwkeurigheid van de positionering op het sternum.



\subsection{Directionele en Abstracte Feedback}
\label{subsec:directionele-abstracte-feedback}

Hoewel een Ghost Avatar de doelpositie aangeeft, zijn aanvullende cues vaak nodig om de gebruiker in real-time te corrigeren wanneer deze van het ideale pad afwijkt.

\begin{itemize}
    \item \textbf{Directionele Feedback:} Deze cues leiden de gebruiker actief naar de juiste positie. \textcite{Diller2024} identificeert hierbij technieken zoals 
    3D-pijlen die de richting van de benodigde kracht aangeven, of 'Rubber Bands' (virtuele elastieken) die de afstand tussen de huidige handpositie en de doelpositie 
    visualiseren. Voor CPR-training zijn pijlen bijzonder nuttig om aan te geven dat een compressie dieper moet zijn of dat de handen volledig ontlast moeten worden 
    (recoil).
    \item \textbf{Abstracte Feedback:} Hierbij wordt informatie versimpeld tot symbolen of kleuren. De meest effectieve methode is \textit{Color Coding}. 
    In een Mixed Reality-omgeving kan de Ghost Avatar of de reanimatiepop groen oplichten bij een correcte uitvoering, en rood verkleuren bij een foutieve 
    diepte of frequentie. Dit biedt onmiddellijke bevestiging zonder dat de gebruiker tekstuele instructies hoeft te lezen.
\end{itemize}

\subsection{Feedforward versus Feedback}
\label{subsec:feedforward-vs-feedback}

Een cruciaal conceptueel onderscheid in motorisch leren is het verschil tussen \textit{feedforward} en \textit{feedback} informatie.

\textcite{Sigrist2013} legt uit dat \textbf{feedback} informatie geeft over een actie die al heeft plaatsgevonden 
(bijv. "de vorige compressie was te ondiep"). \textbf{Feedforward} daarentegen toont de gewenste actie vóór of tijdens de uitvoering 
(bijv. de Ghost Avatar die de beweging voordoet). 

In dit onderzoek wordt gestreefd naar een synergie tussen beide:
\begin{enumerate}
    \item De Ghost Avatar fungeert als een \textbf{feedforward-mechanisme} door continu het ideale pad en tempo aan te geven.
    \item De Color Coding en directionele pijlen fungeren als \textbf{concurrent feedback} door de afwijking van dat ideale pad in real-time te markeren.
\end{enumerate}

Volgens \textcite{Sigrist2013} is deze combinatie in de vroege leerfase van complexe taken essentieel om een correcte 'interne representatie' 
van de beweging op te bouwen, op voorwaarde dat de hulp later geleidelijk wordt afgebouwd om afhankelijkheid te voorkomen (zie \autoref{subsec:guidance-hypothesis}).


\section{Technische Infrastructuur en Communicatie}
\label{sec:technische-infrastructuur}

De betrouwbaarheid van een Mixed Reality-reanimatietrainer valt of staat bij de technische integratie tussen de fysieke pop en de digitale interface. 
In deze sectie worden de protocollen voor datatransmissie, de impact van latentie en de rol van de ontwikkelomgeving toegelicht.

\subsection{Bluetooth Low Energy (BLE) en GATT-profielen}
\label{subsec:ble-gatt}

Voor de draadloze communicatie tussen de reanimatiepop en de Meta Quest 3 is \textit{Bluetooth Low Energy} (BLE) het aangewezen protocol. 
Volgens \textcite{ChaariFourati2020} is BLE bij uitstek geschikt voor medische sensoren vanwege het lage energieverbruik en de geoptimaliseerde 
overdracht van kleine datapakketten.



De pop fungeert binnen dit netwerk als een \textit{GATT-server} (Generic Attribute Profile). De data (zoals compressiediepte in millimeters) 
wordt georganiseerd in \textit{characteristics}, die gegroepeerd zijn onder specifieke \textit{services}. De Mixed Reality-headset treedt op als 
\textit{client} en maakt gebruik van een 'subscribe'-mechanisme. Hierdoor hoeft de headset niet constant om data te vragen (polling), maar verstuurt 
de pop automatisch een update zodra er een verandering in de sensordata optreedt. Dit draagt bij aan een efficiëntere verwerking en een langere batterijduur 
van de apparatuur \autocite{ChaariFourati2020}.

\subsection{Latency en Gebruikerservaring}
\label{subsec:latency-ux}

In een real-time feedbacksysteem is \textit{latency} - de vertraging tussen de fysieke actie en de visuele reactie - 
een kritieke factor voor het motorisch leerproces. Wanneer de visuele cue niet synchroon loopt met de haptische ervaring (het indrukken van de pop), 

ontstaat er een cognitieve dissonantie die de leercurve negatief beïnvloedt.

Volgens het klassieke onderzoek van \textcite{Miller1968} ligt de grens voor een 'onmiddellijke' systeemreactie rond de $100$ ms. 
Boven deze grens ervaart de menselijke hersenpan een merkbare vertraging, wat bij fijn-motorische taken zoals CPR kan leiden tot een over- of 
ondercorrectie van de diepte en frequentie. Voor een optimale Mixed Reality-ervaring is de ambitie echter om de \textit{motion-to-photon latency} 
(de tijd tussen beweging en de update van de pixels in de bril) nog lager te houden, idealiter onder de $20$ ms, om misselijkheid en desoriëntatie te 
voorkomen \autocite{Glover2019}.

\subsection{De Unity Engine en Meta XR SDK}
\label{subsec:unity-meta-sdk}

De softwarematige brug tussen de ruwe sensordata en de \textit{ghost avatar} wordt gevormd door de Unity game engine in combinatie met de \textit{Meta XR Core SDK}. 



Unity dient als de \textit{rendering engine} die de 3D-modellen en visuele feedback in real-time berekent. De \textit{Meta XR SDK} is 
essentieel voor de \textit{spatial awareness} van het systeem. Het biedt de benodigde API's voor:
\begin{itemize}
    \item \textbf{Video Passthrough:} Het streamen van de camerabeelden van de werkelijkheid naar de displays van de headset met minimale vertraging.
    \item \textbf{Spatial Anchoring:} Het vastzetten van het digitale coördinatensysteem op een fysiek punt (de pop), zodat de Ghost Avatar stabiel 
    blijft ongeacht de hoofdbewegingen van de gebruiker \autocite{Glover2019}.
\end{itemize}

Door deze technologieën te combineren, wordt de abstracte data van de BLE-verbinding omgezet in een tastbare, ruimtelijke ervaring die de student helpt 
bij het internaliseren van de correcte reanimatietechniek.