\chapter{\IfLanguageName{dutch}{Stand van zaken}{State of the art}}%
\label{ch:stand-van-zaken}

Dit hoofdstuk biedt een diepgaande analyse van de theoretische en technische fundamenten die noodzakelijk zijn voor de ontwikkeling van dit onderzoek. 
Waar de inleiding de maatschappelijke relevantie van reanimatietraining schetste, spitst dit hoofdstuk zich toe op de medische standaarden, 
de psychologische aspecten van motorisch leren en de technologische mogelijkheden van Mixed Reality. Deze literatuurstudie vormt de wetenschappelijke 
basis voor de methodologie en de technische keuzes die in \autoref{ch:methodologie} worden toegelicht.

\section{Reanimatie: Richtlijnen en Kwaliteitscriteria}
\label{sec:reanimatie-richtlijnen}

Om een effectief feedback-systeem te ontwerpen, moet eerst worden vastgesteld aan welke medische standaarden een kwalitatieve reanimatie moet voldoen. 
De leidraad hiervoor wordt gevormd door de internationale protocollen van de European Resuscitation Council (ERC).

\subsection{De ERC Guidelines 2025}
\label{subsec:erc-guidelines-2025}

De meest recente standaarden voor reanimatie zijn vastgelegd in de \textit{ERC Guidelines 2025 for Adult Basic Life Support}. 
Deze richtlijnen zijn gebaseerd op de wetenschappelijke consensus van de International Liaison Committee on Resuscitation (ILCOR) en 
introduceren enkele cruciale verschuivingen ten opzichte van de voorgaande 2021-editie \autocite{Smyth2025}.

De 2025-richtlijnen leggen een verhoogde nadruk op de snelheid van handelen en de ondersteunende rol van technologie in de overlevingsketen.
 De belangrijkste wijzigingen en aandachtspunten zijn:

\begin{itemize}
    \item \textbf{Onmiddellijk alarmeren (Call First):} In tegenstelling tot eerdere protocollen wordt nu geadviseerd om de noodcentrale direct te bellen bij een niet-reageerbaar slachtoffer, nog vóór de volledige beoordeling van de ademhaling. De ademhalingscontrole vindt plaats terwijl de hulpverlener de dispatcher aan de lijn heeft \autocite{Smyth2025}.
    \item \textbf{Herkenning van agonale ademhaling:} Er is een grotere focus op het herkennen van abnormale ademhalingspatronen zoals "gasping" (happen naar lucht) en "panting" (bij sporters), die vaak foutief als een teken van leven worden geïnterpreteerd \autocite{Smyth2025}.
    \item \textbf{Integratie van technologie:} Voor het eerst worden digitale hulpmiddelen, variërend van video-instructies door dispatchers tot het gebruik van Virtual Reality (VR) en Augmented Reality (AR) in trainingen, expliciet ingebed in de relevante secties van de richtlijnen in plaats van als aparte categorie \autocite{Smyth2025}.
\end{itemize}

Deze verschuiving naar technologische integratie legitimeert de inzet van Mixed Reality als trainingsinstrument. 
Het systeem moet de gebruiker niet alleen instrueren over de techniek, maar ook ondersteunen in de kritieke besluitvorming die door de ERC 2025 wordt benadrukt.

\subsection{Kwaliteitscriteria voor High-Quality CPR (HQ-CPR)}
\label{subsec:kwaliteitscriteria-cpr}

De effectiviteit van reanimatie wordt direct bepaald door de fysiologische impact van de handelingen op het lichaam van het slachtoffer. Om de overlevingskans te maximaliseren, 
hanteert de \textcite{Smyth2025} strikte kwantitatieve criteria voor wat wordt gedefinieerd als 'High-Quality CPR'. 
Voor de ontwikkeling van een Mixed Reality feedback-systeem zijn deze parameters de kernwaarden waarop de algoritmen en visuele cues worden gekalibreerd.

De belangrijkste technische parameters zijn samengevat in \autoref{tab:cpr-metrics}:

\begin{table}[h]
    \centering
    \begin{tabular}{lll}
        \toprule
        \textbf{Parameter} & \textbf{Doelwaarde} & \textbf{Fysiologisch belang} \\
        \midrule
        Compressiediepte & 5 -- 6 cm & Genereren van voldoende bloedstroom \\
        Compressiefrequentie & 100 -- 120 bpm & Optimaliseren van de coronaire perfusiedruk \\
        Borstkas-recoil & Volledige ontlasting & Toelaten van ventriculaire vulling \\
        Handpositie & Onderste helft sternum & Maximale compressie van de ventrikels \\
        \bottomrule
    \end{tabular}
    \caption{Technische parameters voor kwalitatieve reanimatie conform ERC 2025 \autocite{Smyth2025}.}
    \label{tab:cpr-metrics}
\end{table}

\paragraph{Compressiediepte}
De richtlijnen schrijven een diepte voor van minimaal 5 cm, maar niet meer dan 6 cm. Compressies die ondieper zijn dan 5 cm resulteren in een 
inadequate bloedstroom naar de hersenen en het hart. Omgekeerd vergroten compressies dieper dan 6 cm het risico op iatrogeen letsel, zoals ribfracturen of 
interne bloedingen \autocite{Smyth2025}. In een MR-omgeving is dit de meest uitdagende parameter om te visualiseren door het gebrek aan diepteperceptie van bovenaf.

\paragraph{Compressiefrequentie}
Het ideale ritme ligt tussen de 100 en 120 compressies per minuut. Een te lage frequentie leidt tot een te lage gemiddelde perfusiedruk, terwijl een te hoge 
frequentie (boven de 120 bpm) de borstkas onvoldoende tijd geeft voor recoil, wat de efficiëntie per compressie verlaagt.

\paragraph{Recoil en Leunen (Leaning)}
Een veelvoorkomende fout bij hulpverleners is het blijven 'leunen' op de borstkas na een compressie. De ERC benadrukt dat 
volledige \textit{recoil} (terugslag) essentieel is zodat het intrathoracale volume herstelt en het hart zich opnieuw met 
bloed kan vullen voor de volgende cyclus. Feedback-systemen moeten hier specifiek op controleren, aangezien dit een fout is 
die door de hulpverlener vaak niet wordt gevoeld.