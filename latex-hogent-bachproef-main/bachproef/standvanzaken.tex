\chapter{\IfLanguageName{dutch}{Stand van zaken}{State of the art}}%
\label{ch:stand-van-zaken}

Dit hoofdstuk biedt een diepgaande analyse van de theoretische en technische fundamenten die noodzakelijk zijn voor de ontwikkeling van dit onderzoek. 
Waar de inleiding de maatschappelijke relevantie van reanimatietraining schetste, spitst dit hoofdstuk zich toe op de medische standaarden, 
de psychologische aspecten van motorisch leren en de technologische mogelijkheden van Mixed Reality. Deze literatuurstudie vormt de wetenschappelijke 
basis voor de methodologie en de technische keuzes die in \autoref{ch:methodologie} worden toegelicht.

\section{Reanimatie: Richtlijnen en Kwaliteitscriteria}
\label{sec:reanimatie-richtlijnen}

Om een effectief feedback-systeem te ontwerpen, moet eerst worden vastgesteld aan welke medische standaarden een kwalitatieve reanimatie moet voldoen. 
De leidraad hiervoor wordt gevormd door de internationale protocollen van de European Resuscitation Council (ERC).

\subsection{De ERC Guidelines 2025}
\label{subsec:erc-guidelines-2025}

De meest recente standaarden voor reanimatie zijn vastgelegd in de \textit{ERC Guidelines 2025 for Adult Basic Life Support}. 
Deze richtlijnen zijn gebaseerd op de wetenschappelijke consensus van de International Liaison Committee on Resuscitation (ILCOR) en 
introduceren enkele cruciale verschuivingen ten opzichte van de voorgaande 2021-editie \autocite{Smyth2025}.

De 2025-richtlijnen leggen een verhoogde nadruk op de snelheid van handelen en de ondersteunende rol van technologie in de overlevingsketen.
 De belangrijkste wijzigingen en aandachtspunten zijn:

\begin{itemize}
    \item \textbf{Onmiddellijk alarmeren (Call First):} In tegenstelling tot eerdere protocollen wordt nu geadviseerd om de noodcentrale direct te bellen bij een niet-reageerbaar slachtoffer, nog vóór de volledige beoordeling van de ademhaling. De ademhalingscontrole vindt plaats terwijl de hulpverlener de dispatcher aan de lijn heeft \autocite{Smyth2025}.
    \item \textbf{Herkenning van agonale ademhaling:} Er is een grotere focus op het herkennen van abnormale ademhalingspatronen zoals "gasping" (happen naar lucht) en "panting" (bij sporters), die vaak foutief als een teken van leven worden geïnterpreteerd \autocite{Smyth2025}.
    \item \textbf{Integratie van technologie:} Voor het eerst worden digitale hulpmiddelen, variërend van video-instructies door dispatchers tot het gebruik van Virtual Reality (VR) en Augmented Reality (AR) in trainingen, expliciet ingebed in de relevante secties van de richtlijnen in plaats van als aparte categorie \autocite{Smyth2025}.
\end{itemize}

Deze verschuiving naar technologische integratie legitimeert de inzet van Mixed Reality als trainingsinstrument. 
Het systeem moet de gebruiker niet alleen instrueren over de techniek, maar ook ondersteunen in de kritieke besluitvorming die door de ERC 2025 wordt benadrukt.

\subsection{Kwaliteitscriteria voor High-Quality CPR (HQ-CPR)}
\label{subsec:kwaliteitscriteria-cpr}

De effectiviteit van reanimatie wordt direct bepaald door de fysiologische impact van de handelingen op het lichaam van het slachtoffer. Om de overlevingskans te maximaliseren, 
hanteert de \textcite{Smyth2025} strikte kwantitatieve criteria voor wat wordt gedefinieerd als 'High-Quality CPR'. 
Voor de ontwikkeling van een Mixed Reality feedback-systeem zijn deze parameters de kernwaarden waarop de algoritmen en visuele cues worden gekalibreerd.

De belangrijkste technische parameters zijn samengevat in \autoref{tab:cpr-metrics}:

\begin{table}[h]
    \centering
    \begin{tabular}{lll}
        \toprule
        \textbf{Parameter} & \textbf{Doelwaarde} & \textbf{Fysiologisch belang} \\
        \midrule
        Compressiediepte & 5 -- 6 cm & Genereren van voldoende bloedstroom \\
        Compressiefrequentie & 100 -- 120 bpm & Optimaliseren van de coronaire perfusiedruk \\
        Borstkas-recoil & Volledige ontlasting & Toelaten van ventriculaire vulling \\
        Handpositie & Onderste helft sternum & Maximale compressie van de ventrikels \\
        \bottomrule
    \end{tabular}
    \caption{Technische parameters voor kwalitatieve reanimatie conform ERC 2025 \autocite{Smyth2025}.}
    \label{tab:cpr-metrics}
\end{table}

\paragraph{Compressiediepte}
De richtlijnen schrijven een diepte voor van minimaal 5 cm, maar niet meer dan 6 cm. Compressies die ondieper zijn dan 5 cm resulteren in een 
inadequate bloedstroom naar de hersenen en het hart. Omgekeerd vergroten compressies dieper dan 6 cm het risico op iatrogeen letsel, zoals ribfracturen of 
interne bloedingen \autocite{Smyth2025}. In een MR-omgeving is dit de meest uitdagende parameter om te visualiseren door het gebrek aan diepteperceptie van bovenaf.

\paragraph{Compressiefrequentie}
Het ideale ritme ligt tussen de 100 en 120 compressies per minuut. Een te lage frequentie leidt tot een te lage gemiddelde perfusiedruk, terwijl een te hoge 
frequentie (boven de 120 bpm) de borstkas onvoldoende tijd geeft voor recoil, wat de efficiëntie per compressie verlaagt.

\paragraph{Recoil en Leunen (Leaning)}
Een veelvoorkomende fout bij hulpverleners is het blijven 'leunen' op de borstkas na een compressie. De ERC benadrukt dat 
volledige \textit{recoil} (terugslag) essentieel is zodat het intrathoracale volume herstelt en het hart zich opnieuw met 
bloed kan vullen voor de volgende cyclus. Feedback-systemen moeten hier specifiek op controleren, aangezien dit een fout is 
die door de hulpverlener vaak niet wordt gevoeld.

\subsection{Manikin-gebaseerde Feedback: Laerdal QCPR}
\label{subsec:laerdal-qcpr}

De huidige standaard voor objectieve feedback in reanimatietraining wordt gevormd door intelligente oefenpoppen, waarvan de Laerdal 
QCPR-technologie de meest wijdverspreide is. Dit systeem maakt gebruik van een draadloze verbinding (vaak Bluetooth) tussen sensoren in 
de pop en externe software op een monitor of tablet \autocite{Cortegiani2017}.

\paragraph{Werking van de sensoren}
De sensoren in de pop meten continu de mechanische belasting op de borstkas. De belangrijkste gemeten parameters zijn de compressiediepte, 
de frequentie, de handpositie en de mate van volledige ontlasting (\textit{recoil}). Volgens \textcite{Cortegiani2017} berekent de software op 
basis van deze data een zogenaamde 'compression score': een gewogen gemiddelde (0--100\%) dat de algehele kwaliteit van de reanimatie uitdrukt. 
In hun onderzoek bij middelbare scholieren bleek dat studenten die trainden met deze real-time feedback een significant hogere score behaalden (mediaan 90\%) 
dan studenten die enkel vertrouwden op de instructies van een menselijke expert (mediaan 67\%) \autocite{Cortegiani2017}.

\begin{figure}[H] 
    \centering
    \includegraphics[width=0.8\textwidth]{../graphics/littleAnnCPR} 
    \caption{De Laerdal Little Anne QCPR pop in combinatie met de feedback-app op een tablet \autocite{Cortegiani2017}.}
    \label{fig:littleann-cpr}
\end{figure}

\paragraph{Superieure herkenning van technische fouten}
Een inzicht uit de studie van \textcite{Cortegiani2017} is dat softwarematige feedback vooral effectief is 
voor parameters die voor het menselijk oog moeilijk waarneembaar zijn. Zo presteerde de QCPR-groep significant beter op het 
gebied van \textit{chest recoil} (71\% correcte uitvoering versus 24\% in de controlegroep). Dit suggereert dat menselijke instructeurs 
moeite hebben om minieme fouten in de ontlasting van de borstkas te zien, terwijl de sensoren in de pop dit feilloos registreren. 

Hoewel dit systeem de technische vaardigheidsverwerving versnelt, blijft de interface beperkt tot een 2D-weergave op een extern scherm. 
Zoals in de volgende sectie wordt toegelicht, creëert dit een nieuwe uitdaging op het gebied van aandacht en cognitieve belasting.

\subsection{De beperkingen van 2D-displays en het Split-Attention Effect}
\label{subsec:2d-beperkingen}

Hoewel systemen zoals Laerdal QCPR objectieve data leveren, dwingen ze de gebruiker tot een onnatuurlijke werkhouding waarbij de aandacht verdeeld moet worden tussen de fysieke handeling en een extern scherm. Recent onderzoek van \textcite{Hurt2024} werpt een nieuw licht op de cognitieve nadelen van deze methode.

\paragraph{Situational Awareness en Performance}
Een cruciale bevinding in het onderzoek naar motorische taken is de sterke positieve correlatie tussen 
\textit{situational awareness} (SA) en prestatie bij het gebruik van Mixed Reality. \textcite{Hurt2024} stelt dat het 
beperken van \textit{split-attention} essentieel is: wanneer informatie direct in het gezichtsveld wordt geprojecteerd, 
kunnen deelnemers hun focus volledig op de taak houden, wat leidt tot betere resultaten. Opvallend genoeg ontbrak deze correlatie 
bij het gebruik van tablets. Dit suggereert dat de tablet-modaliteit de gebruiker dwingt om de aandacht zo onregelmatig te verdelen dat 
de natuurlijke link tussen omgevingsbewustzijn en prestatie wordt verbroken \autocite{Hurt2024}.

\paragraph{De Familiarity Bias en het TAM2-model}
Een belangrijk punt voor dit onderzoek is waarom tablets momenteel de standaard zijn, ondanks de cognitieve nadelen. 
\textcite{Hurt2024} verklaart dit aan de hand van het \textit{Technology Acceptance Model 2} (TAM2). Deelnemers rapporteren vaak een voorkeur voor tablets, 
niet omdat ze efficiënter zijn, maar vanwege een \textit{familiarity bias}. Omdat studenten in hun dagelijks leven constant tablets gebruiken voor studie en 
ontspanning, ervaren ze een lagere drempel in gebruiksgemak (\textit{perceived ease of use}), ongeacht de werkelijke taakbelasting \autocite{Hurt2024}. 
Bij complexe, taakgerichte handelingen zoals CPR kan deze gewenning echter de technische tekortkomingen van de modaliteit maskeren.

\paragraph{Task Load en Visuele Continuïteit}
Aanvullend onderzoek toont aan dat de \textit{task load} — een maatstaf voor mentale, fysieke en temporele inspanning — significant lager is bij 
Augmented Reality-instructies dan bij instructies op een monitor of tablet \autocite[Tang et al. in]{Hurt2024}. 
Door informatie via een \textit{heads-up display} (HUD) aan te bieden, hoeft de gebruiker zijn ogen vrijwel nooit van de taak af te wenden. 
Dit vermindert niet alleen de oog- en hoofdbewegingen, maar zorgt er ook voor dat informatie "in context" blijft \autocite[Haines et al. in]{Hurt2024}. 
Voor reanimatietraining betekent dit dat de overgang van het observeren van de feedback naar het uitvoeren van de compressie naadloos verloopt, in tegenstelling 
tot de gefragmenteerde ervaring bij 2D-schermen.


\subsection{De Guidance Hypothesis en de Ontwikkeling van Interne Representaties}
\label{subsec:guidance-hypothesis}

Een fundamenteel risico bij het gebruik van real-time feedbacksystemen is de \textit{guidance hypothesis}. 
Volgens \textcite{Sigrist2013} kan permanente externe feedback tijdens het leerproces leiden tot een afhankelijkheid 
van deze informatiebron, wat de ontwikkeling van een autonoom spiergeheugen belemmert.

\paragraph{Het mechanisme van afhankelijkheid}
De kern van de \textit{guidance hypothesis} is dat de student de externe feedback (bijv. een balkje op een scherm) gebruikt als een kruk. 
In plaats van te leren vertrouwen op \textit{intrinsic feedback}, de interne zintuiglijke waarneming van de eigen lichaamshouding en krachtoefening, 
reageert de student puur op de \textit{augmented feedback} van het systeem \autocite{Sigrist2013}. Dit heeft tot gevolg dat de prestaties tijdens de training 
uitstekend zijn, maar dat de vaardigheid onmiddellijk degradeert zodra de feedback wordt weggenomen (\textit{retention tests}). 
De student heeft namelijk geen "interne bewegingsrepresentatie" opgebouwd die zonder hulp kan worden opgeroepen.



\paragraph{Simpele versus complexe motorische taken}
\textcite{Sigrist2013} maken een belangrijk onderscheid tussen simpele en complexe taken. Bij simpele taken is constante feedback vrijwel altijd schadelijk voor het 
leerproces op lange termijn. Echter, bij complexe taken zoals reanimatie, waarbij houding, diepte, tempo en recoil simultaan moeten worden beheerst, 
kan real-time feedback juist helpen om de cognitieve belasting te verlagen in de vroege leerfase. Het helpt de student om de "structuur" van de beweging te begrijpen 
zonder overweldigd te raken.

\paragraph{Strategieën voor effectief leren}
Om de nadelen van de \textit{guidance hypothesis} te beperken, stelt de literatuur verschillende strategieën voor die relevant zijn voor het ontwerp van een 
Mixed Reality PoC:
\begin{itemize}
    \item \textbf{Fading feedback:} De frequentie van de visuele cues moet afnemen naarmate het vaardigheidsniveau van de student toeneemt \autocite{Sigrist2013}.
    \item \textbf{No-feedback trials:} Het inbouwen van oefensessies zonder hologrammen is essentieel om te controleren of de student de correcte diepte en frequentie op gevoel kan reproduceren.
    \item \textbf{Bandwidth feedback:} Hierbij wordt alleen feedback gegeven wanneer de foutmarge een bepaalde drempel overschrijdt, wat de student dwingt om binnen de 
    "veilige zone" zelfstandig te corrigeren.
\end{itemize}

Door de \textit{ghost avatar} in Mixed Reality niet constant, maar adaptief aan te bieden, kan het systeem de student begeleiden in de vroege fase zonder dat er een 
permanente afhankelijkheid ontstaat die de inzetbaarheid in een echte noodsituatie ondermijnt.