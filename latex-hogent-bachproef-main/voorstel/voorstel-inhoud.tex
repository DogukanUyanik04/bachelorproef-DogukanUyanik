%---------- Inleiding ---------------------------------------------------------

% TODO: Is dit voorstel gebaseerd op een paper van Research Methods die je
% vorig jaar hebt ingediend? Heb je daarbij eventueel samengewerkt met een
% andere student?
% Zo ja, haal dan de tekst hieronder uit commentaar en pas aan.

%\paragraph{Opmerking}

% Dit voorstel is gebaseerd op het onderzoeksvoorstel dat werd geschreven in het
% kader van het vak Research Methods dat ik (vorig/dit) academiejaar heb
% uitgewerkt (met medesturent VOORNAAM NAAM als mede-auteur).
% 

\section{Inleiding}%
\label{sec:inleiding}

\subsection{Kaderen Thema en Doelgroep}
Een hartstilstand is wereldwijd één van de meest voorkomende doodsoorzaken buiten het ziekenhuis. 
De overlevingskans van een slachtoffer hangt in grote mate af van de kwaliteit van de reanimatie die in de eerste minuten wordt toegepast. 
Het correct aanleren van deze vaardigheid aan zorgverleners is daarom cruciaal.

De doelgroep van dit onderzoek bestaat specifiek uit bachelorstudenten Verpleegkunde en Geneeskunde die reanimatietechnieken moeten beoefenen.

\subsection{Probleemstelling}
In de huidige onderwijspraktijk, zoals bij de opleiding Verpleegkunde en partners zoals UZ Gent, wordt vaak gebruikgemaakt van digitale oefenpoppen (zoals de Laerdal Little Anne). 
Deze poppen sturen data via Bluetooth naar een extern scherm of tablet (bijv. een iPad), waarop de student grafieken en scores kan aflezen. Hoewel deze technologie een stap vooruit is in vergelijking met oude fysieke mannequins, 
brengt deze oplossing twee concrete problemen met zich mee die het leerproces verstoren.

Ten eerste is er sprake van abstracte feedback. De tablet toont de prestaties van de student vaak in de vorm van percentages, grafieken of algemene scores. 
Voor een student die een motorische vaardigheid aanleert, is deze data lastig te vertalen naar een fysieke correctie. 
De student ziet op het scherm wellicht een score van 60\%, maar krijgt geen directe, ruimtelijke instructie over hoe de 
handpositie of drukhoek fysiek aangepast moet worden om die score te verhogen.

Ten tweede veroorzaakt het gebruik van een extern scherm het zogenaamde \textit{split-attention effect}. 
Om feedback te ontvangen wordt de student genoodzaakt om de blik af te wenden van de 'patiënt' (de pop) naar de tablet. 
Dit verbreekt niet alleen de concentratie, maar is ook niet representatief voor een realistische noodsituatie, waarbij de focus continu op het slachtoffer moet liggen.

Deze problematiek staat in contrast met de bredere technologische evolutie binnen de zorgsector. 
De afgelopen jaren heeft Augmented Reality (AR) en Mixed Reality (MR) steeds vaker zijn weg gevonden naar de klinische praktijk, 
zoals bij chirurgische visualisaties of anatomische lessen. De meerwaarde van het verrijken van de werkelijkheid is in andere domeinen reeds aangetoond, 
wat een logische basis vormt om dit toe te passen op reanimatietraining.
\subsection{Centrale onderzoeksvraag}
Op basis van de bovenstaande probleemstelling wordt de volgende centrale onderzoeksvraag geformuleerd:

\begin{quote}
    \textit{In welke mate kan een Mixed Reality-applicatie de kwaliteit van feedback tijdens reanimatietraining verbeteren in vergelijking met de huidige tablet-gebaseerde methoden?}
\end{quote}

\subsection{Deelvragen}
Om een antwoord te geven op deze hoofdvraag, wordt het onderzoek gesplitst in meerdere deelaspecten. 
Eerst wordt gefocust op het \textbf{probleemdomein} om de tekortkomingen van de huidige situatie in kaart te brengen:

\begin{itemize}
    \item Welke veelvoorkomende reanimatiefouten worden door de huidige hardware niet of onvoldoende gedetecteerd tijdens een training?
    \item Hoe ervaren studenten de vertaalslag van de huidige abstracte vorm van feedback (zoals grafieken, cijfers en scores) naar de fysieke correctie van hun handeling?
    \item Welke specifieke feedback-parameters van hartcompressie worden door studenten als het meest problematisch of onduidelijk ervaren op de huidige dashboards?
\end{itemize}

Vervolgens richt het onderzoek zich op het \textbf{oplossingsdomein}, waarbij de focus ligt op de technische ontwikkeling en validatie van de applicatie:

\begin{itemize}
    \item Hoe kan de sensordata van de oefenpop real-time uitgelezen en verwerkt worden binnen een Mixed Reality-omgeving?
    \item Op welke manier kan feedback visueel worden weergegeven zodat de student zijn vaardigheid autonoom kan verbeteren?
    \item In welke mate stelt de ontwikkelde Proof of Concept (PoC) studenten in staat om fouten sneller en accurater te corrigeren in vergelijking met de traditionele tablet-feedback?
\end{itemize}

\subsection{Onderzoeksdoelstelling}
De tekortkomingen van de huidige methodiek creëren een duidelijke opportuniteit voor het inzetten van deze nieuwe opkomende technologie, Mixed Reality. 
Het doel van deze bachelorproef is het ontwikkelen van een \textbf{Proof of Concept (PoC)} om de potentie van AR/MR binnen dit domein te valideren. 
Centraal staat de vraag of het integreren van real-time feedback in het gezichtsveld van de student leidt tot een betere gebruikerservaring en een efficiënter leerproces.

Het concrete eindresultaat van dit onderzoek bestaat uit twee delen: ten eerste een werkend prototype dat communiceert met de oefenpop en visuele feedback projecteert; 
ten tweede een validatierapport waarin de effectiviteit van deze oplossing wordt vergeleken met de traditionele trainingsmethoden.

%---------- Stand van zaken ---------------------------------------------------

\section{Literatuurstudie}%
\label{sec:literatuurstudie}

Hier beschrijf je de \emph{state-of-the-art} rondom je gekozen onderzoeksdomein, d.w.z.\ een inleidende, doorlopende tekst over het onderzoeksdomein van je bachelorproef. Je steunt daarbij heel sterk op de professionele \emph{vakliteratuur}, en niet zozeer op populariserende teksten voor een breed publiek. Wat is de huidige stand van zaken in dit domein, en wat zijn nog eventuele open vragen (die misschien de aanleiding waren tot je onderzoeksvraag!)?

Je mag de titel van deze sectie ook aanpassen (literatuurstudie, stand van zaken, enz.). Zijn er al gelijkaardige onderzoeken gevoerd? Wat concluderen ze? Wat is het verschil met jouw onderzoek?

Verwijs bij elke introductie van een term of bewering over het domein naar de vakliteratuur, bijvoorbeeld~\autocite{Hykes2013}! Denk zeker goed na welke werken je refereert en waarom.

Draag zorg voor correcte literatuurverwijzingen! Een bronvermelding hoort thuis \emph{binnen} de zin waar je je op die bron baseert, dus niet er buiten! Maak meteen een verwijzing als je gebruik maakt van een bron. Doe dit dus \emph{niet} aan het einde van een lange paragraaf. Baseer nooit teveel aansluitende tekst op eenzelfde bron.

Als je informatie over bronnen verzamelt in JabRef, zorg er dan voor dat alle nodige info aanwezig is om de bron terug te vinden (zoals uitvoerig besproken in de lessen Research Methods).

% Voor literatuurverwijzingen zijn er twee belangrijke commando's:
% \autocite{KEY} => (Auteur, jaartal) Gebruik dit als de naam van de auteur
%   geen onderdeel is van de zin.
% \textcite{KEY} => Auteur (jaartal)  Gebruik dit als de auteursnaam wel een
%   functie heeft in de zin (bv. ``Uit onderzoek door Doll & Hill (1954) bleek
%   ...'')

Je mag deze sectie nog verder onderverdelen in subsecties als dit de structuur van de tekst kan verduidelijken.

%---------- Methodologie ------------------------------------------------------
\section{Methodologie}%
\label{sec:methodologie}

Hier beschrijf je hoe je van plan bent het onderzoek te voeren. Welke onderzoekstechniek ga je toepassen om elk van je onderzoeksvragen te beantwoorden? Gebruik je hiervoor literatuurstudie, interviews met belanghebbenden (bv.~voor requirements-analyse), experimenten, simulaties, vergelijkende studie, risico-analyse, PoC, \ldots?

Valt je onderwerp onder één van de typische soorten bachelorproeven die besproken zijn in de lessen Research Methods (bv.\ vergelijkende studie of risico-analyse)? Zorg er dan ook voor dat we duidelijk de verschillende stappen terug vinden die we verwachten in dit soort onderzoek!

Vermijd onderzoekstechnieken die geen objectieve, meetbare resultaten kunnen opleveren. Enquêtes, bijvoorbeeld, zijn voor een bachelorproef informatica meestal \textbf{niet geschikt}. De antwoorden zijn eerder meningen dan feiten en in de praktijk blijkt het ook bijzonder moeilijk om voldoende respondenten te vinden. Studenten die een enquête willen voeren, hebben meestal ook geen goede definitie van de populatie, waardoor ook niet kan aangetoond worden dat eventuele resultaten representatief zijn.

Uit dit onderdeel moet duidelijk naar voor komen dat je bachelorproef ook technisch voldoen\-de diepgang zal bevatten. Het zou niet kloppen als een bachelorproef informatica ook door bv.\ een student marketing zou kunnen uitgevoerd worden.

Je beschrijft ook al welke tools (hardware, software, diensten, \ldots) je denkt hiervoor te gebruiken of te ontwikkelen.

Probeer ook een tijdschatting te maken. Hoe lang zal je met elke fase van je onderzoek bezig zijn en wat zijn de concrete \emph{deliverables} in elke fase?

%---------- Verwachte resultaten ----------------------------------------------
\section{Verwacht resultaat, conclusie}%
\label{sec:verwachte_resultaten}

Hier beschrijf je welke resultaten je verwacht. Als je metingen en simulaties uitvoert, kan je hier al mock-ups maken van de grafieken samen met de verwachte conclusies. Benoem zeker al je assen en de onderdelen van de grafiek die je gaat gebruiken. Dit zorgt ervoor dat je concreet weet welk soort data je moet verzamelen en hoe je die moet meten.

Wat heeft de doelgroep van je onderzoek aan het resultaat? Op welke manier zorgt jouw bachelorproef voor een meerwaarde?

Hier beschrijf je wat je verwacht uit je onderzoek, met de motivatie waarom. Het is \textbf{niet} erg indien uit je onderzoek andere resultaten en conclusies vloeien dan dat je hier beschrijft: het is dan juist interessant om te onderzoeken waarom jouw hypothesen niet overeenkomen met de resultaten.

