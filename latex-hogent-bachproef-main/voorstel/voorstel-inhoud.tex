%---------- Inleiding ---------------------------------------------------------

\section{Inleiding}%
\label{sec:inleiding}

\subsection{Kaderen Thema en Doelgroep}
Een hartstilstand is wereldwijd één van de meest voorkomende doodsoorzaken buiten het ziekenhuis. 
De overlevingskans van een slachtoffer hangt in grote mate af van de kwaliteit van de reanimatie die in de eerste minuten wordt toegepast, wat dus betekent dat het 
correct aanleren van deze vaardigheid aan zorgverleners is daarom cruciaal \autocite{Olasveengen2021}.

De doelgroep van dit onderzoek bestaat specifiek uit bachelorstudenten Verpleegkunde en Geneeskunde die reanimatietechnieken moeten beoefenen.

\subsection{Probleemstelling}
In de huidige onderwijspraktijk, zoals bij de opleiding Verpleegkunde en partners zoals UZ Gent, wordt vaak gebruikgemaakt van digitale oefenpoppen (zoals de Laerdal Little Anne). 
Deze poppen sturen data via Bluetooth naar een extern scherm of tablet (bijv. een iPad), waarop de student grafieken en scores kan aflezen. Hoewel deze technologie een stap vooruit is in vergelijking met oude fysieke mannequins, 
brengt deze oplossing twee concrete problemen met zich mee die het leerproces verstoren.

Ten eerste is er sprake van abstracte feedback. De tablet toont de prestaties van de student vaak in de vorm van percentages, grafieken of algemene scores. 
Voor een student die een motorische vaardigheid aanleert, is deze data lastig te vertalen naar een fysieke correctie \autocite{Sigrist2013}. 
De student ziet op het scherm wellicht een score van 60\%, maar krijgt geen directe, ruimtelijke instructie over hoe de 
handpositie of drukhoek fysiek aangepast moet worden om die score te verhogen.

Ten tweede veroorzaakt het gebruik van een extern scherm het zogenaamde \textit{split-attention effect} \autocite{Sigrist2013}. 
Om feedback te ontvangen wordt de student genoodzaakt om de blik af te wenden van de 'patiënt' (de pop) naar de tablet. 
Dit verbreekt niet alleen de concentratie, maar is ook niet representatief voor een realistische noodsituatie, waarbij de focus continu op het slachtoffer moet liggen.

Deze problematiek staat in contrast met de bredere technologische evolutie binnen de zorgsector. 
De afgelopen jaren heeft Augmented Reality (AR) en Mixed Reality (MR) steeds vaker zijn weg gevonden naar de klinische praktijk, 
zoals bij chirurgische visualisaties of anatomische lessen\autocite{DAngelo2021}. De meerwaarde van het verrijken van de werkelijkheid is in andere domeinen reeds aangetoond, 
wat een logische basis vormt om dit toe te passen op reanimatietraining.
\subsection{Centrale onderzoeksvraag}
Op basis van de bovenstaande probleemstelling wordt de volgende centrale onderzoeksvraag geformuleerd:

\begin{quote}
    \textit{In welke mate kan een Mixed Reality-applicatie de kwaliteit van feedback tijdens reanimatietraining verbeteren in vergelijking met de huidige tablet-gebaseerde methoden?}
\end{quote}

\subsection{Deelvragen}
Om een antwoord te geven op deze hoofdvraag, wordt het onderzoek gesplitst in meerdere deelaspecten. 
Eerst wordt gefocust op het \textbf{probleemdomein} om de tekortkomingen van de huidige situatie in kaart te brengen:

\begin{itemize}
    \item Welke veelvoorkomende reanimatiefouten worden door de huidige hardware niet of onvoldoende gedetecteerd tijdens een training?
    \item Hoe ervaren studenten de vertaalslag van de huidige abstracte vorm van feedback (zoals grafieken, cijfers en scores) naar de fysieke correctie van hun handeling?
    \item Welke specifieke feedback-parameters van hartcompressie worden door studenten als het meest problematisch of onduidelijk ervaren op de huidige dashboards?
\end{itemize}

Vervolgens richt het onderzoek zich op het \textbf{oplossingsdomein}, waarbij de focus ligt op de technische ontwikkeling en validatie van de applicatie:

\begin{itemize}
    \item Hoe kan de sensordata van de oefenpop real-time uitgelezen en verwerkt worden binnen een Mixed Reality-omgeving?
    \item Op welke manier kan feedback visueel worden weergegeven zodat de student zijn vaardigheid autonoom kan verbeteren?
    \item In welke mate stelt de ontwikkelde Proof of Concept (PoC) studenten in staat om fouten sneller en accurater te corrigeren in vergelijking met de traditionele tablet-feedback?
\end{itemize}

\subsection{Onderzoeksdoelstelling}
De tekortkomingen van de huidige methodiek creëren een duidelijke opportuniteit voor het inzetten van deze nieuwe opkomende technologie, Mixed Reality. 
Het doel van deze bachelorproef is het ontwikkelen van een \textbf{Proof of Concept (PoC)} om de potentie van AR/MR binnen dit domein te valideren. 
Centraal staat de vraag of het integreren van real-time feedback in het gezichtsveld van de student leidt tot een betere gebruikerservaring en een efficiënter leerproces.

Het concrete eindresultaat van dit onderzoek bestaat uit twee delen: ten eerste een werkend prototype dat communiceert met de oefenpop en visuele feedback projecteert; 
ten tweede een validatierapport waarin de effectiviteit van deze oplossing wordt vergeleken met de traditionele trainingsmethoden.

%---------- Stand van zaken ---------------------------------------------------

\section{Literatuurstudie}
\label{sec:literatuurstudie}


\subsection{Kwaliteitscriteria voor Reanimatie (CPR)}
De doeltreffenheid van reanimatie wordt bepaald door de nauwkeurigheid van de uitgevoerde handelingen. 
In Europa gebruikt men de \textit{European Resuscitation Council} (ERC) om zo een standaard vast te leggen omtrent reanimatie.
In de meest recente richtlijnen benadrukken \textcite{Olasveengen2021} dat 'High-Quality CPR' essentieel is voor de overlevingskans van het slachtoffer.


Voor de ontwikkeling van een feedback-systeem is het van groot belang dus om te specifiëren welke parameters nodig zijn om de kwaliteit van compressies te beoordelen:
\begin{itemize}
    \item \textbf{Compressiediepte:} De borstkas moet ingedrukt worden met een diepte van minimaal 5 cm en maximaal 6 cm. Compressies die ondieper zijn, 
    genereren onvoldoende bloedstroom, terwijl diepere compressies het risico op letsel schade vergroten.
    \item \textbf{Compressiefrequentie (Rate):} Het ritme van de compressies moet tussen de 100 en 120 slagen per minuut liggen.
\item \textbf{Recoil (Leunen):} Na elke compressie moet de borstkas volledig de kans krijgen om terug te veren ('recoil'). 
Het is cruciaal dat de hulpverlener niet op de borstkas blijft leunen, zodat het hart zich opnieuw kan vullen met bloed.
    \item \textbf{Handpositie:} De handen dienen geplaatst te worden op de onderste helft van het sternum (borstbeen), in het midden van de borstkas.
\end{itemize}

Daarnaast schrijven de richtlijnen een verhouding voor van 30 compressies afgewisseld met 2 beademingen (30:2), 
tenzij de hulpverlener niet in staat is om te beademen; in dat geval wordt continue hartmassage aanbevolen~\autocite{Olasveengen2021}.


\subsection{Beperkingen van Traditionele Feedbackmethoden}
De Laerdal QCPR-technologie representeert de huidige standaard voor objectieve feedback in reanimatietraining. 
Het systeem maakt gebruik van geïntegreerde sensoren in de oefenpop die via Bluetooth verbinding maken met externe software. 
Tijdens de reanimatie meet het systeem continu parameters zoals compressiediepte, frequentie en de mate van borstkas-recoil. 
Deze data wordt real-time gevisualiseerd op een gekoppeld scherm, waardoor studenten hun motoriek direct kunnen corrigeren op basis van kwantitatieve gegevens.
De effectiviteit van deze methode wordt ondersteund door \textcite{Cortegiani2017}, die aantoonden dat studenten die trainden met deze real-time QCPR-feedback 
significant hogere compressiescores en betere technische vaardigheden behaalden dan studenten die enkel vertrouwden op verbale correcties van een instructeur.

Er is echter een belangrijk nadeel aan het huidige onderzoek naar feedback.\textcite{Sigrist2013} geven aan dat veel studies gebaseerd zijn op hele simpele taken in een lab.
Maar reanimeren is juist een heel complex vaardigheid waarbij je je hele lichaam moet gebruiken. 
Het is dus een grote vraag of de resultaten van die simpele testen wel gelden voor het echte werk. Hierdoor duikt het zogezegde 
'Guidance Hypothesis' tevoorschijn: studenten zullen zich puur focussen op het groen houden van de grafieken en worden afhankelijk van een scherm, 
in plaats van dat ze moeten aanvoelen hoeveel kracht ze werkelijk moeten zetten \autocite{Sigrist2013}. Als het visuele hulpmiddel wegvalt missen ze dus 
het benodigde spiergeheugen, hierdoor maken de auteurs de stelling dat simpele 2D-grafieken vaak tekortschieten en in de toekomst vervangen worden door 
realistische simulaties binnen virtuele omgevingen om dit gat te dichten \autocite{Sigrist2013}.


\subsection{Cognitieve Belasting en het Split-Attention Effect}
Recent onderzoek benadrukt de fysieke impact van display-modaliteit op de cognitieve belasting \autocite{Hurt2024}. In een studie naar motorische taken werd een signicant
positieve correlatie gevonden tussen situational awareness en taakprestatie \autocite{Hurt2024}. Bij traditionele methodes met externe schermen moet de student echter steeds
wegkijken, waardoor deze focus wordt verbroken. 
Hoewel proefpersonen tablets soms als prettig ervaarden, komt dit volgens de auteur vooral door gewenning (familiarity bias) en niet door efficiëntie \autocite{Hurt2024}.
Hurt concludeert dat Mixed Reality een betere oplossing is voor taken waarbij je je handen gebruikt. Informatie kan hierbij direct in het zicht worden geplaatst,
waardoor waardoor het split-attention effect wordt voorkomen \autocite{Hurt2024}.

\subsection{Mixed Reality in de Gezondheidszorg}
Volgens \textcite{Gerup2020} is de betekenis van Mixed Reality (MR) het samensmelten van de werkelijkheid en de virtuele wereld. De technologie
combineert zowel Virtual Reality (VR) als Augmented Reality (AR) technologie \autocite{Gerup2020}. 
Het resultaat hiervan is een nieuwe omgeving waarin objecten van
de reële wereld en de virtuele wereld op hetzelfde moment met elkaar kunnen interacteren \autocite{Carroll2020}.

De opkomst van MR in de zorg antwoord meteen op de tekortkomingen van eerdere simulatiemethoden. \textcite{DAngelo2021} beschrijven dat pure Virtual Reality vaak
tekortschiet door het gebrak aan haptische feedback (aanraking stimulatie), terwijl fysieke poppen juist de mogelijkheid missen om prestaties objectief te meten. 
MR biedt hier een oplossing voor, het combineert de tastbare weerstand van een fysiek model met de digitale analyse van een computer. Hierdoor kunnen
motorische vaardigheiden trainen op een fysieke object, terwijl de software direct meetbare feedback geeft over hun handelen \autocite{DAngelo2021}.
Wel merken de auteurs op dat veel validatiestudies nog kleinschalig zijn, waardoor groter onderzoek nodig is om de klinische impact volledig te bevestigen.

\subsection{Visualisatietechnieken voor Motorische Vaardigheden}
Om motorische vaardigheden effectief te ontwikkelen via Mixed Reality, is de keuze van visuele cue van groot belang. In een recente survey van 39 studies categoriseren 
\textcite{Diller2024} de verschillende feedbackmethoden in drie hoofdgroepen.

Ten eerste is er sprake van \textbf{positionele feedback}, waarbij het doel is dat de participant een specifieke houding aanneemt. Meest gebruikte techniek hiervoor is de Transparent
Target Avatar of ook wel de 'Ghost': een doorzichtig virtueel model dat diens als spiegelbeeld voor de ideale houding \autocite{Diller2024}.

Ten tweede onderscheidt men \textbf{directionele feedback}. Hier is de bedoeling de gebruiker naar de juiste positie te leiden. Dit kan variëren van eenvoudige 3D-pijlen tot complexere technieken
zoals Rubber Bands (virtuele elastieken die de afstand tot het doel visualiseren) of Trajectories (lichtsporen die het bewegingspad aangeven) \autocite{Diller2024}. 

Tot slot is er \textbf{abstracte feedback}, waarbij informatie zoveel mogelijk wordt versimpeld. Methodiek dat veel voorkomt is Color Coding, waarbij ledematen van de virtuele avatar rood of groen
kleuren afhankelijk van de nauwkeurigheid van de uitvoering \autocite{Diller2024}. Uit de analyse van Diller blijkt dat positionele feedback (zoals de avatar) momenteel het meest wordt toegepast, omdat deze manier
de gebruiker de meest complete informatie geeft over de lichaamsoriëntatie.


\subsection{Technische Aspecten: Connectiviteit en Latency}
De technische communicatie tussen de reanimatiepop en de MR-bril verloopt via Bluetooth Low Energy (BLE). Dit protocol is ideaal voor medische toepassingen door zijn lage energieverbruik.
Het werkt via de GATT-profiel, waarbij de pop data-pakketten (zoals compressiediepte) verstuurt naar de bril \autocite{ChaariFourati2020}. Een kritische factor hierbij is latency:
de vertraging tussen actie en feedback moet onder de 100ms blijven zodat de gebruiker de interactie als 'direct' ervaart \autocite{ChaariFourati2020}.
Aangezien BLE een snelheid van minimaal 7,5ms aankan is het technisch mogelijk om in deze grens te blijven.

Voor de visualisatie en interactie wordt gebruiktgemaakt van de Unity game engine. Aangezien de Meta Quest gebruikmaakt van Video Passthrough-technologie, dient Unity als de brug tussen
de camerabeelden van de echte wereld en de virtuele feedback \autocite{Raymer2023}. Via de Meta XR Core SDK worden de beelden van de omgeving real-time ingeladen, waarna Unity de 
BLE-data dat binnenkomt van de reanimatiepop vertaalt naar virtuele overlays \autocite{Glover2019}. De engine zorgt ervoor dat de hologrammen stabiel op de videofeed worden geprojecteerd 'anchoring', 
zodat de gebruiker de illusie krijgt dat de feedback zich fysiek in de ruimte bevindt \autocite{Glover2019}. 

%---------- Methodologie ------------------------------------------------------
\section{Methodologie}%
\label{sec:methodologie}

Hier beschrijf je hoe je van plan bent het onderzoek te voeren. Welke onderzoekstechniek ga je toepassen om elk van je onderzoeksvragen te beantwoorden? Gebruik je hiervoor literatuurstudie, interviews met belanghebbenden (bv.~voor requirements-analyse), experimenten, simulaties, vergelijkende studie, risico-analyse, PoC, \ldots?

Valt je onderwerp onder één van de typische soorten bachelorproeven die besproken zijn in de lessen Research Methods (bv.\ vergelijkende studie of risico-analyse)? Zorg er dan ook voor dat we duidelijk de verschillende stappen terug vinden die we verwachten in dit soort onderzoek!

Vermijd onderzoekstechnieken die geen objectieve, meetbare resultaten kunnen opleveren. Enquêtes, bijvoorbeeld, zijn voor een bachelorproef informatica meestal \textbf{niet geschikt}. De antwoorden zijn eerder meningen dan feiten en in de praktijk blijkt het ook bijzonder moeilijk om voldoende respondenten te vinden. Studenten die een enquête willen voeren, hebben meestal ook geen goede definitie van de populatie, waardoor ook niet kan aangetoond worden dat eventuele resultaten representatief zijn.

Uit dit onderdeel moet duidelijk naar voor komen dat je bachelorproef ook technisch voldoen\-de diepgang zal bevatten. Het zou niet kloppen als een bachelorproef informatica ook door bv.\ een student marketing zou kunnen uitgevoerd worden.

Je beschrijft ook al welke tools (hardware, software, diensten, \ldots) je denkt hiervoor te gebruiken of te ontwikkelen.

Probeer ook een tijdschatting te maken. Hoe lang zal je met elke fase van je onderzoek bezig zijn en wat zijn de concrete \emph{deliverables} in elke fase?

%---------- Verwachte resultaten ----------------------------------------------
\section{Verwacht resultaat, conclusie}%
\label{sec:verwachte_resultaten}

Hier beschrijf je welke resultaten je verwacht. Als je metingen en simulaties uitvoert, kan je hier al mock-ups maken van de grafieken samen met de verwachte conclusies. Benoem zeker al je assen en de onderdelen van de grafiek die je gaat gebruiken. Dit zorgt ervoor dat je concreet weet welk soort data je moet verzamelen en hoe je die moet meten.

Wat heeft de doelgroep van je onderzoek aan het resultaat? Op welke manier zorgt jouw bachelorproef voor een meerwaarde?

Hier beschrijf je wat je verwacht uit je onderzoek, met de motivatie waarom. Het is \textbf{niet} erg indien uit je onderzoek andere resultaten en conclusies vloeien dan dat je hier beschrijft: het is dan juist interessant om te onderzoeken waarom jouw hypothesen niet overeenkomen met de resultaten.

